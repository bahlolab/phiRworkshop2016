\documentclass[13pt,aspectratio=169]{beamer}\usepackage[]{graphicx}\usepackage[]{color}
%% maxwidth is the original width if it is less than linewidth
%% otherwise use linewidth (to make sure the graphics do not exceed the margin)
\makeatletter
\def\maxwidth{ %
  \ifdim\Gin@nat@width>\linewidth
    \linewidth
  \else
    \Gin@nat@width
  \fi
}
\makeatother

\definecolor{fgcolor}{rgb}{0.345, 0.345, 0.345}
\newcommand{\hlnum}[1]{\textcolor[rgb]{0.686,0.059,0.569}{#1}}%
\newcommand{\hlstr}[1]{\textcolor[rgb]{0.192,0.494,0.8}{#1}}%
\newcommand{\hlcom}[1]{\textcolor[rgb]{0.678,0.584,0.686}{\textit{#1}}}%
\newcommand{\hlopt}[1]{\textcolor[rgb]{0,0,0}{#1}}%
\newcommand{\hlstd}[1]{\textcolor[rgb]{0.345,0.345,0.345}{#1}}%
\newcommand{\hlkwa}[1]{\textcolor[rgb]{0.161,0.373,0.58}{\textbf{#1}}}%
\newcommand{\hlkwb}[1]{\textcolor[rgb]{0.69,0.353,0.396}{#1}}%
\newcommand{\hlkwc}[1]{\textcolor[rgb]{0.333,0.667,0.333}{#1}}%
\newcommand{\hlkwd}[1]{\textcolor[rgb]{0.737,0.353,0.396}{\textbf{#1}}}%

\usepackage{framed}
\makeatletter
\newenvironment{kframe}{%
 \def\at@end@of@kframe{}%
 \ifinner\ifhmode%
  \def\at@end@of@kframe{\end{minipage}}%
  \begin{minipage}{\columnwidth}%
 \fi\fi%
 \def\FrameCommand##1{\hskip\@totalleftmargin \hskip-\fboxsep
 \colorbox{shadecolor}{##1}\hskip-\fboxsep
     % There is no \\@totalrightmargin, so:
     \hskip-\linewidth \hskip-\@totalleftmargin \hskip\columnwidth}%
 \MakeFramed {\advance\hsize-\width
   \@totalleftmargin\z@ \linewidth\hsize
   \@setminipage}}%
 {\par\unskip\endMakeFramed%
 \at@end@of@kframe}
\makeatother

\definecolor{shadecolor}{rgb}{.97, .97, .97}
\definecolor{messagecolor}{rgb}{0, 0, 0}
\definecolor{warningcolor}{rgb}{1, 0, 1}
\definecolor{errorcolor}{rgb}{1, 0, 0}
\newenvironment{knitrout}{}{} % an empty environment to be redefined in TeX

\usepackage{alltt}
\mode<presentation>{\usetheme{default}}

%%% macros
\newcommand{\todo}[1]{\textcolor{red}{@TODO: #1}}
\newcommand{\tc}[2]{\textcolor{#1}{#2}}
\renewcommand{\tt}[1]{\texttt{#1}}
\newcommand{\ig}{\includegraphics}

%%% packages
\usepackage{tikz}
\hypersetup{colorlinks,linkcolor=,urlcolor=blue}

%%% Environments
% make knitr output denser
\ifdefined\knitrout
\renewenvironment{knitrout}{\setlength{\topsep}{0mm}}{}
\else
\fi

%%% Colours
\definecolor{green}{RGB}{0, 180, 0} % green is too light
\definecolor{Blue}{RGB}{0, 51, 102} % dark navy blue
%\definecolor{Blue}{RGB}{0, 0, 255} % other blue

%%% Beamer elements
%% Templates
% page number
\setbeamertemplate{footline}{%
    \raisebox{5pt}{\makebox[\paperwidth]{\hfill\makebox[20pt]{\color{Blue}
		\scriptsize\insertframenumber}}}\hspace*{5pt}}

% frametitle
\setbeamertemplate{frametitle}{\LARGE\color{Blue}\textbf{\insertframetitle}\vskip-1pt}
\setbeamertemplate{itemize items}[ball]
% \setbeamertemplate{itemize/enumerate body begin}{\Large}
% \setbeamertemplate{itemize/enumerate subbody begin}{\normalsize}
\beamertemplatenavigationsymbolsempty % clean out all the crap

%% fonts
\setbeamerfont{title}{size=\Huge,series=\bfseries}

%% colours
\setbeamercolor{title}{fg=Blue}
\setbeamercolor*{item}{fg=Blue}

%%% Other
\title{The basics of R}
\author{Peter Diakumis - Bahlo Lab \\[2pt] R Workshop 1}
\date{\scriptsize{\today}}
\graphicspath{{../figs/}}
\IfFileExists{upquote.sty}{\usepackage{upquote}}{}
\begin{document}

\begin{frame}[fragile]
    \titlepage

\end{frame}

{ % all template changes are local to this group.
    \begin{frame}[plain]
	\begin{tikzpicture}[remember picture,overlay]
	    \node[at=(current page.center)] {
		\includegraphics[width=\paperwidth]{iceberg}
	    };
	\end{tikzpicture}
    \end{frame}
}

\begin{frame}
    \frametitle{What can we learn today?}
    \begin{columns}[t]
	\column{0.2\textwidth}
	\begin{itemize}
	    \item What is R
	\end{itemize}
	\vskip14pt
	\includegraphics[width=.9\textwidth]{rapp_logo}\\
	\column{0.5\textwidth}
	\begin{itemize}
	    \item How (and where) to find help, resouRces, courses, tutorials
	\end{itemize}
	\includegraphics[width=.8\textwidth,trim={0 0cm 0 17cm},clip]{google}\\
	\includegraphics[width=.8\textwidth]{stackoverflow}\\
	\includegraphics[width=.3\textwidth]{coursera}\hspace*{10mm}
	\includegraphics[width=.3\textwidth]{edx}
	\column{0.3\textwidth}
	\begin{itemize}
	    \item How to write and run R code \& scripts
	\end{itemize}
	\vskip14pt
	\includegraphics[width=1.0\textwidth]{rstudio_logo}
    \end{columns}
\end{frame}

\begin{frame}
    \frametitle{What can we learn today?}
    \begin{columns}[t]
	\column{0.3\textwidth}
	\begin{itemize}
	    \item \textbf{numbers, characters \& logicals} in R
	\end{itemize}
	\vskip14pt
	\includegraphics[width=.9\textwidth]{acgt}
	\column{0.3\textwidth}
	\begin{itemize}
	    \item \textbf{vectors} in R
	\end{itemize}
	\vskip14pt
	\includegraphics[width=.7\textwidth]{dna}
	\column{0.4\textwidth}
	\begin{itemize}
	    \item \textit{simple} \textbf{functions} in R
	\end{itemize}
	\vskip14pt
	\includegraphics[width=.9\textwidth]{central-dogma}
    \end{columns}
\end{frame}

\begin{frame}
    \frametitle{What can we learn today?}
    \begin{columns}[t]
	\column{0.4\textwidth}
	\begin{itemize}
	    \item \textbf{matrix, data.frame \& list} in R
		\alert{(\textit{not today..})}
	\end{itemize}
	\vskip14pt
	\includegraphics[width=1.0\textwidth]{protein}
	\column{0.6\textwidth}
	\begin{itemize}
	    \item \textbf{packages} in R
		\alert{(\textit{a little bit..})}
	\end{itemize}
	\vskip14pt
	% trim from left anti-clockwise
	\includegraphics[width=.9\textwidth,trim={0 2cm 0 2.5cm},clip]{cell}
    \end{columns}
\end{frame}

\begin{frame}
    \begin{center}
	\tc{Blue}{\textbf{\Huge{Part 1}}}
    \end{center}
\end{frame}

\begin{frame}
    \frametitle{Stuff about R}
	\begin{itemize}
	    \item Programming language for statisticians \& data scientists
		created in early 90s by \textbf{R}oss Ihaka \&
		\textbf{R}obert Gentleman in Auckland, NZ
	    \item Statistical computing, advanced methodologies,
		amazing visualisations - \textbf{all in one}, and
		\textbf{free}ly available (you didn't pay anything for
		installation, did you?)
	    \item Easy to \textbf{learn}. Hard to \textbf{master}.
	\end{itemize}
	\includegraphics[width=.2\textwidth]{oz_map}
	\includegraphics[width=.2\textwidth]{heatmap}
	\includegraphics[width=.2\textwidth]{boxplot}
	\includegraphics[width=.2\textwidth]{barplot}
	\includegraphics[width=.2\textwidth]{scatterplot}
\end{frame}

\begin{frame}
    \frametitle{Stuff about R}
    \begin{columns}
	\column{0.4\textwidth}
	\includegraphics[height=.6\textheight]{cran}
	\column{0.6\textwidth}
	\begin{itemize}
	    \item Close to 9,000 \textbf{free} R packages (extensions)
		available (i.e. if you're analysing some kind of data, someone
		has probably created a helpful package \& workflow)
		\vskip10pt
	    \item Excellent for reproducibility of analyses \& dry
		lab experiments
		\vskip10pt
	    \item Easy collaboration and scientific report generation
		\textit{(maybe HackR 3 or 4?)}
	\end{itemize}
	\end{columns}
\end{frame}

\begin{frame}
    \frametitle{Start out...}
    \begin{center}
	\includegraphics[height=.9\textheight]{no-idea}
    \end{center}
\end{frame}

\begin{frame}
    \frametitle{...Train Hard}
    \includegraphics[height=.9\textheight]{bear}
\end{frame}

\begin{frame}
    \frametitle{Then...}
    \includegraphics[height=.9\textheight]{cat_karate}
\end{frame}

\begin{frame}
    \begin{center}
	\tc{Blue}{\textbf{\Huge{Very very short demo}}}
    \end{center}
\end{frame}

\begin{frame}[fragile]
\begin{knitrout}\small
\definecolor{shadecolor}{rgb}{0.969, 0.969, 0.969}\color{fgcolor}\begin{kframe}
\begin{alltt}
\hlstd{> }\hlkwd{install.packages}\hlstd{(}\hlstr{"ggplot2"}\hlstd{)}
\hlstd{> }\hlkwd{install.packages}\hlstd{(}\hlstr{"oz"}\hlstd{)}
\end{alltt}
\end{kframe}
\end{knitrout}

\begin{knitrout}\small
\definecolor{shadecolor}{rgb}{0.969, 0.969, 0.969}\color{fgcolor}\begin{kframe}
\begin{alltt}
\hlstd{> }\hlkwd{library}\hlstd{(ggplot2);} \hlkwd{library}\hlstd{(oz)}
\hlstd{> }\hlkwd{load}\hlstd{(}\hlstr{"../data/ozdata.rda"}\hlstd{)}
\hlstd{> }\hlkwd{head}\hlstd{(ozdata)}
\end{alltt}
\begin{verbatim}
      long       lat group order state border
1 129.0114 -31.57795     1     1    WA  coast
2 128.7262 -31.69202     1     2    WA  coast
3 128.5932 -31.78707     1     3    WA  coast
4 128.4981 -31.80608     1     4    WA  coast
5 128.3840 -31.88213     1     5    WA  coast
6 128.1369 -31.93916     1     6    WA  coast
\end{verbatim}
\end{kframe}
\end{knitrout}
\end{frame}

\begin{frame}[fragile]
\begin{knitrout}\small
\definecolor{shadecolor}{rgb}{0.969, 0.969, 0.969}\color{fgcolor}\begin{kframe}
\begin{alltt}
\hlstd{> }\hlcom{# map showing all states in different colours}
\hlstd{> }\hlstd{p} \hlkwb{<-} \hlkwd{qplot}\hlstd{(long, lat,} \hlkwc{data} \hlstd{= ozdata,} \hlkwc{geom} \hlstd{=} \hlstr{"polygon"}\hlstd{,}
\hlstd{+ }      \hlkwc{fill} \hlstd{= state,} \hlkwc{main} \hlstd{=} \hlstr{"States of Australia"}\hlstd{)} \hlopt{+} \hlkwd{coord_equal}\hlstd{()}
\end{alltt}
\end{kframe}
\end{knitrout}
\end{frame}

\begin{frame}[fragile]
\begin{knitrout}\small
\definecolor{shadecolor}{rgb}{0.969, 0.969, 0.969}\color{fgcolor}
\includegraphics[width=\maxwidth]{figure/unnamed-chunk-5-1} 

\end{knitrout}
\end{frame}

\begin{frame}
    \frametitle{R comes in 3 basic flavours on a Mac:}
    \includegraphics[width=\textwidth]{flavours}
	    \vskip10pt
    \textbf{\LARGE{\tc{Blue}{RStudio - R app - R terminal}}}
\end{frame}

\begin{frame}
    \frametitle{RStudio is awRsome!}
    \begin{center}
	\includegraphics[height=.9\textheight]{rstudio}
    \end{center}
\end{frame}

\begin{frame}
    \begin{center}
	\tc{Blue}{\textbf{\Huge{Let's start breaking R!}}}
		\vskip10pt
	\LARGE{... but first let's open Microsoft Excel}
    \end{center}
\end{frame}

\begin{frame}
    \begin{center}
	\frametitle{Entering data in Excel is easy.
	    \\But how do we enter data into R?}
	\includegraphics[height=.6\textheight]{excel1}
    \end{center}
\end{frame}

\begin{frame}
    \frametitle{Writing R Code - 2 ways}
    \begin{center}
	\includegraphics[height=.8\textheight]{yoda}
    \end{center}
\end{frame}

\begin{frame}
    \begin{columns}[b]
	\column{0.5\textwidth}
	\textbf{\tc{Blue}{\LARGE{1. Interactively in the console (\tt{>} is
		    the \tc{green}{prompt})}}}
	\vskip10pt
	\includegraphics[height=.7\textheight]{rstudio_console}
	\column{0.5\textwidth}
	\textbf{\tc{Blue}{\LARGE{2. In a simple .R file \\
		    (Cmd+Enter = Magic!)}}}
	\vskip10pt
	\includegraphics[height=.7\textheight]{rstudio_source}
    \end{columns}
\end{frame}

\begin{frame}[fragile]
    \frametitle{R Code Style Basics}
    \tc{Blue}{\textbf{Assign values to variables
	    \vskip10pt
	    Use \tt{<-}}
	\textit{(less than minus with \alert{no spaces} in between!!)}}.\\
    Don't use \alert{\tt{=}}.\\ Because I said so.
\begin{knitrout}\small
\definecolor{shadecolor}{rgb}{0.969, 0.969, 0.969}\color{fgcolor}\begin{kframe}
\begin{alltt}
\hlcom{# Good}
\hlstd{x} \hlkwb{<-} \hlnum{5}
\hlstd{my_name} \hlkwb{<-} \hlstr{"Yoda"}

\hlcom{# Bad (but works...)}
\hlstd{x} \hlkwb{=} \hlnum{5}
\hlstd{my_name} \hlkwb{=} \hlstr{"Yoda"}
\end{alltt}
\end{kframe}
\end{knitrout}
\begin{knitrout}\small
\definecolor{shadecolor}{rgb}{0.969, 0.969, 0.969}\color{fgcolor}\begin{kframe}
\begin{alltt}
\hlstd{x} \hlkwb{<-} \hlnum{5} \hlcom{# Good}
\hlstd{x} \hlopt{< -} \hlnum{5} \hlcom{# Bad (works, but not as you'd expect!)}
\end{alltt}
\begin{verbatim}
[1] FALSE
\end{verbatim}
\end{kframe}
\end{knitrout}

\end{frame}

\begin{frame}[fragile]
    \frametitle{R Code Style Basics}
    \textbf{\tc{Blue}{Variable names}}
    \begin{itemize}
	\item Don't start with a number, and no spaces!
	\item Keep it simple but descriptive
	\item Names are like sensitive passwords:
	    \tt{NAME}, \tt{Name}, \tt{name} \& \tt{NaMe} are distinct!
    \end{itemize}
\begin{knitrout}\small
\definecolor{shadecolor}{rgb}{0.969, 0.969, 0.969}\color{fgcolor}\begin{kframe}
\begin{alltt}
\hlcom{# Good}
day_one
day_1
dayOne
day.one

\hlcom{# Bad}
day one \hlcom{# won't work with spaces}
1_day \hlcom{# won't work starting with a number}
first_day_of_the_year_two_thousand_fifteen \hlcom{# very long}
fdotyttf \hlcom{# OMG LOL}
\end{alltt}
\end{kframe}
\end{knitrout}
\end{frame}

\begin{frame}[fragile]
    \frametitle{R Code Style Basics}
    \textbf{\tc{Blue}{Spacing = Clarity!}}
	    \vskip10pt
	    Try reading out your code. Put spaces between things like you'd
	    normally do when writing normal text.
	    \vskip10pt
	    Don't worry too much - will learn with experience :-)
\begin{knitrout}\small
\definecolor{shadecolor}{rgb}{0.969, 0.969, 0.969}\color{fgcolor}\begin{kframe}
\begin{alltt}
\hlstd{> }\hlcom{# Whichonewouldyouprefertoreadat2:35am?}
\hlstd{> }\hlstd{x} \hlkwb{<-} \hlstd{(y} \hlopt{+} \hlstd{z)} \hlopt{*} \hlnum{12} \hlopt{^} \hlstd{(a} \hlopt{-} \hlstd{b)} \hlopt{/} \hlkwd{sum}\hlstd{(k, l)}
\hlstd{> }\hlstd{x}\hlkwb{<-}\hlstd{(y}\hlopt{+}\hlstd{z)}\hlopt{*}\hlnum{12}\hlopt{^}\hlstd{(a}\hlopt{-}\hlstd{b)}\hlopt{/}\hlkwd{sum}\hlstd{(k,l)}
\end{alltt}
\end{kframe}
\end{knitrout}

Check out \href{https://google-styleguide.googlecode.com/svn/trunk/Rguide.xml}{web\_link1}
and \href{http://adv-r.had.co.nz/Style.html}{web\_link2} if you are curious.
\end{frame}

\begin{frame}[fragile]
    \frametitle{R Code Style Basics}
    \textbf{\tc{Blue}{Comment your code with \#} -
	everything on the right
	of \tt{\#} is ignored}
    \begin{itemize}
	\item Either in a \alert{separate line}...
    \end{itemize}

\begin{knitrout}\small
\definecolor{shadecolor}{rgb}{0.969, 0.969, 0.969}\color{fgcolor}\begin{kframe}
\begin{alltt}
\hlstd{> }\hlcom{# Create variables height and width}
\hlstd{> }\hlstd{height} \hlkwb{<-} \hlnum{3}
\hlstd{> }\hlstd{width} \hlkwb{<-} \hlnum{6}
\end{alltt}
\end{kframe}
\end{knitrout}
    \begin{itemize}
	\item Or in the \alert{same line} ...
    \end{itemize}

\begin{knitrout}\small
\definecolor{shadecolor}{rgb}{0.969, 0.969, 0.969}\color{fgcolor}\begin{kframe}
\begin{alltt}
\hlstd{> }\hlstd{area} \hlkwb{<-} \hlstd{height} \hlopt{*} \hlstd{width}
\hlstd{> }\hlstd{area} \hlcom{# should be 18}
\end{alltt}
\begin{verbatim}
[1] 18
\end{verbatim}
\begin{alltt}
\hlstd{> }\hlcom{# area * 2}
\end{alltt}
\end{kframe}
\end{knitrout}

\end{frame}

\begin{frame}[fragile]
    \frametitle{Variables }
    \textbf{\tc{Blue}{Why do we need ``variables''?}}
    \vskip10pt
    In the code below, ``Yoda'' is written 3 times. (\tt{paste()} simply sticks
    everything inside it together and prints the result)

\begin{knitrout}\small
\definecolor{shadecolor}{rgb}{0.969, 0.969, 0.969}\color{fgcolor}\begin{kframe}
\begin{alltt}
\hlstd{> }\hlkwd{paste}\hlstd{(}\hlstr{"Yoda knows everything"}\hlstd{)}
\end{alltt}
\begin{verbatim}
[1] "Yoda knows everything"
\end{verbatim}
\begin{alltt}
\hlstd{> }\hlkwd{paste}\hlstd{(}\hlstr{"Luke learnt everything from Yoda"}\hlstd{)}
\end{alltt}
\begin{verbatim}
[1] "Luke learnt everything from Yoda"
\end{verbatim}
\begin{alltt}
\hlstd{> }\hlkwd{paste}\hlstd{(}\hlstr{"I think Yoda is 785 years old"}\hlstd{)}
\end{alltt}
\begin{verbatim}
[1] "I think Yoda is 785 years old"
\end{verbatim}
\end{kframe}
\end{knitrout}

\end{frame}

\begin{frame}[fragile]
    \frametitle{Variables}
    \textbf{\tc{Blue}{Why do we need ``variables''?}}
    \vskip10pt
    Let's say Yoda sends the R script to Peter.\\
    In the code below, ``Peter'' is written \textit{almost} 3 times.

\begin{knitrout}\small
\definecolor{shadecolor}{rgb}{0.969, 0.969, 0.969}\color{fgcolor}\begin{kframe}
\begin{alltt}
\hlstd{> }\hlkwd{paste}\hlstd{(}\hlstr{"Peter knows everything"}\hlstd{)}
\end{alltt}
\begin{verbatim}
[1] "Peter knows everything"
\end{verbatim}
\begin{alltt}
\hlstd{> }\hlkwd{paste}\hlstd{(}\hlstr{"Luke learnt everything from Peter"}\hlstd{)}
\end{alltt}
\begin{verbatim}
[1] "Luke learnt everything from Peter"
\end{verbatim}
\begin{alltt}
\hlstd{> }\hlkwd{paste}\hlstd{(}\hlstr{"I think Pter is 785 years old"}\hlstd{)}
\end{alltt}
\begin{verbatim}
[1] "I think Pter is 785 years old"
\end{verbatim}
\end{kframe}
\end{knitrout}
\end{frame}

\begin{frame}[fragile]
    \frametitle{Variables}
    \textbf{\tc{Blue}{Why do we need ``variables''?}}
    \vskip10pt
    Instead of going through all the code and changing all the
    occurrences of ``Yoda'' to ``Peter'', we can instead create a
    variable e.g. \tt{the\_name} and assign it the value that we
    wish. This way, ``Han'', ``Chewie'', ``Leia'' etc. only need
    to change that variable \alert{once}. This saves time, energy \& eyeballs.
    So variables = good.

\begin{knitrout}\small
\definecolor{shadecolor}{rgb}{0.969, 0.969, 0.969}\color{fgcolor}\begin{kframe}
\begin{alltt}
\hlstd{> }\hlstd{the_name} \hlkwb{<-} \hlstr{"Peter"}
\hlstd{> }\hlkwd{paste}\hlstd{(the_name,} \hlstr{"knows everything"}\hlstd{)}
\end{alltt}
\begin{verbatim}
[1] "Peter knows everything"
\end{verbatim}
\begin{alltt}
\hlstd{> }\hlkwd{paste}\hlstd{(}\hlstr{"Luke learnt everything from"}\hlstd{, the_name)}
\end{alltt}
\begin{verbatim}
[1] "Luke learnt everything from Peter"
\end{verbatim}
\begin{alltt}
\hlstd{> }\hlkwd{paste}\hlstd{(}\hlstr{"I think"}\hlstd{,  the_name,} \hlstr{"is 785 years old"}\hlstd{)}
\end{alltt}
\begin{verbatim}
[1] "I think Peter is 785 years old"
\end{verbatim}
\end{kframe}
\end{knitrout}
\end{frame}


\begin{frame}
    \begin{center}
	\tc{Blue}{\textbf{\Huge{Part 2}}}
    \end{center}
\end{frame}

\begin{frame}[fragile]
    \frametitle{Functions (very very briefly)}
    \begin{columns}[t]
	\column{0.4\textwidth}
	\includegraphics[width=.8\textwidth]{function}
	\column{0.6\textwidth}
	\includegraphics[width=.6\textwidth]{sum_function_excel}\\
\begin{knitrout}\small
\definecolor{shadecolor}{rgb}{0.969, 0.969, 0.969}\color{fgcolor}\begin{kframe}
\begin{alltt}
\hlstd{> }\hlkwd{sum}\hlstd{(}\hlnum{123}\hlstd{,} \hlnum{35}\hlstd{,} \hlnum{59}\hlstd{,} \hlnum{46}\hlstd{,} \hlnum{26}\hlstd{,} \hlnum{245.6}\hlstd{,} \hlnum{25}\hlstd{)}
\end{alltt}
\begin{verbatim}
[1] 559.6
\end{verbatim}
\end{kframe}
\end{knitrout}
    \end{columns}
\end{frame}

\begin{frame}[fragile]
    \frametitle{Functions (very very briefly)}
    \begin{itemize}
	\item The things you put between parentheses are called arguments
	\item Use ``?'' to check out function details (e.g. what type of
	    arguments you can specify)
    \end{itemize}
    \begin{columns}[t]
	\column{0.5\textwidth}
\begin{knitrout}\small
\definecolor{shadecolor}{rgb}{0.969, 0.969, 0.969}\color{fgcolor}\begin{kframe}
\begin{alltt}
\hlstd{> }\hlkwd{sum}\hlstd{(}\hlnum{10}\hlstd{,} \hlnum{20}\hlstd{,} \hlnum{30}\hlstd{,} \hlnum{100}\hlstd{)}
\end{alltt}
\begin{verbatim}
[1] 160
\end{verbatim}
\begin{alltt}
\hlstd{> }\hlkwd{prod}\hlstd{(}\hlnum{1}\hlstd{,} \hlnum{2}\hlstd{,} \hlnum{3}\hlstd{,} \hlnum{4}\hlstd{)}
\end{alltt}
\begin{verbatim}
[1] 24
\end{verbatim}
\end{kframe}
\end{knitrout}

\begin{knitrout}\small
\definecolor{shadecolor}{rgb}{0.969, 0.969, 0.969}\color{fgcolor}\begin{kframe}
\begin{alltt}
\hlstd{> }\hlopt{?}\hlstd{sum}
\hlstd{> }\hlopt{?}\hlstd{prod}
\hlstd{> }\hlopt{?}\hlstd{paste}
\end{alltt}
\end{kframe}
\end{knitrout}
	\column{0.5\textwidth}
\begin{knitrout}\small
\definecolor{shadecolor}{rgb}{0.969, 0.969, 0.969}\color{fgcolor}\begin{kframe}
\begin{alltt}
\hlstd{> }\hlkwd{paste}\hlstd{(}\hlstr{"Star"}\hlstd{,} \hlstr{"Wars:"}\hlstd{,} \hlstr{"Episode VI"}\hlstd{)}
\end{alltt}
\begin{verbatim}
[1] "Star Wars: Episode VI"
\end{verbatim}
\begin{alltt}
\hlstd{> }\hlkwd{paste}\hlstd{(}\hlstr{"Return of"}\hlstd{,} \hlstr{"the Jedi"}\hlstd{)}
\end{alltt}
\begin{verbatim}
[1] "Return of the Jedi"
\end{verbatim}
\end{kframe}
\end{knitrout}
    \end{columns}
\end{frame}

\begin{frame}[fragile]
    \frametitle{Workspace (or Environment)}
    How can I see which variables have been created?
    Just list them with \tt{ls()}:

\begin{knitrout}\small
\definecolor{shadecolor}{rgb}{0.969, 0.969, 0.969}\color{fgcolor}\begin{kframe}
\begin{alltt}
\hlstd{> }\hlkwd{ls}\hlstd{()}
\end{alltt}
\begin{verbatim}
[1] "area"     "height"   "ozdata"   "p"        "the_name" "width"   
[7] "x"       
\end{verbatim}
\begin{alltt}
\hlstd{> }\hlstd{height}
\end{alltt}
\begin{verbatim}
[1] 3
\end{verbatim}
\begin{alltt}
\hlstd{> }\hlkwd{rm}\hlstd{(height)} \hlcom{# rm removes stuff}
\hlstd{> }\hlstd{height}
\end{alltt}


{\ttfamily\noindent\bfseries\color{errorcolor}{Error in eval(expr, envir, enclos): object 'height' not found}}\end{kframe}
\end{knitrout}

\end{frame}

\begin{frame}[fragile]
    \frametitle{Workspace (or Environment)}
    ...or check out the Environment window:
    \vskip14pt
    \includegraphics[width=.8\textwidth]{environment}
\end{frame}

\begin{frame}
    \begin{center}
	\textbf{\Huge{\tc{Blue}{Basic Data Types}}}
    \end{center}
\end{frame}

\begin{frame}[fragile]
    \frametitle{numeric}
    \begin{columns}[b]
	\column{0.3\textwidth}
    All numbers:
    positive, negative, decimals.
    Note how \tt{;} can be used to terminate an expression, instead of using a
    separate line. Neat.
\begin{knitrout}\small
\definecolor{shadecolor}{rgb}{0.969, 0.969, 0.969}\color{fgcolor}\begin{kframe}
\begin{alltt}
\hlstd{> }\hlstd{a} \hlkwb{<-} \hlnum{1}\hlstd{; a}
\end{alltt}
\begin{verbatim}
[1] 1
\end{verbatim}
\begin{alltt}
\hlstd{> }\hlstd{b} \hlkwb{<-} \hlnum{2}\hlstd{; b}
\end{alltt}
\begin{verbatim}
[1] 2
\end{verbatim}
\begin{alltt}
\hlstd{> }\hlstd{c} \hlkwb{<-} \hlnum{1000}\hlstd{; c}
\end{alltt}
\begin{verbatim}
[1] 1000
\end{verbatim}
\end{kframe}
\end{knitrout}
	\column{0.3\textwidth}
\begin{knitrout}\small
\definecolor{shadecolor}{rgb}{0.969, 0.969, 0.969}\color{fgcolor}\begin{kframe}
\begin{alltt}
\hlstd{> }\hlstd{d} \hlkwb{<-} \hlopt{-}\hlnum{34}\hlstd{; d}
\end{alltt}
\begin{verbatim}
[1] -34
\end{verbatim}
\begin{alltt}
\hlstd{> }\hlstd{e} \hlkwb{<-} \hlnum{0}\hlstd{; e}
\end{alltt}
\begin{verbatim}
[1] 0
\end{verbatim}
\begin{alltt}
\hlstd{> }\hlstd{f} \hlkwb{<-} \hlopt{-}\hlnum{753000}\hlstd{; f}
\end{alltt}
\begin{verbatim}
[1] -753000
\end{verbatim}
\end{kframe}
\end{knitrout}
	\column{0.5\textwidth}
    \includegraphics[width=.6\textwidth]{pi_day}
    \vskip6pt
\begin{knitrout}\small
\definecolor{shadecolor}{rgb}{0.969, 0.969, 0.969}\color{fgcolor}\begin{kframe}
\begin{alltt}
\hlstd{> }\hlstd{g} \hlkwb{<-} \hlnum{0.005}\hlstd{; g}
\end{alltt}
\begin{verbatim}
[1] 0.005
\end{verbatim}
\begin{alltt}
\hlstd{> }\hlstd{h} \hlkwb{<-} \hlnum{3.14}\hlstd{; h}
\end{alltt}
\begin{verbatim}
[1] 3.14
\end{verbatim}
\begin{alltt}
\hlstd{> }\hlstd{i} \hlkwb{<-} \hlopt{-}\hlnum{42.9}\hlstd{; i}
\end{alltt}
\begin{verbatim}
[1] -42.9
\end{verbatim}
\end{kframe}
\end{knitrout}
    \end{columns}
\end{frame}

\begin{frame}[fragile]
    \frametitle{character}
    Anything you put in between quotes \tt{``''} \hspace*{10mm}
    \includegraphics[width=.3\textwidth]{dog_ears}
\begin{knitrout}\small
\definecolor{shadecolor}{rgb}{0.969, 0.969, 0.969}\color{fgcolor}\begin{kframe}
\begin{alltt}
\hlstd{> }\hlstd{lucky_num} \hlkwb{<-} \hlstr{"My lucky number is 42"}
\hlstd{> }\hlstd{lucky_num}
\end{alltt}
\begin{verbatim}
[1] "My lucky number is 42"
\end{verbatim}
\begin{alltt}
\hlstd{> }\hlstd{num_but_char} \hlkwb{<-} \hlstr{"13"}
\hlstd{> }\hlstd{num_but_char}
\end{alltt}
\begin{verbatim}
[1] "13"
\end{verbatim}
\begin{alltt}
\hlstd{> }\hlstd{get_published} \hlkwb{<-} \hlstr{"If p-value < 0.05 then Nature awaits..."}
\hlstd{> }\hlstd{get_published}
\end{alltt}
\begin{verbatim}
[1] "If p-value < 0.05 then Nature awaits..."
\end{verbatim}
\end{kframe}
\end{knitrout}
\end{frame}


\begin{frame}[fragile]
    \frametitle{logical}
    \begin{columns}[t]
	\column{.35\textwidth}
	Things are either TRUE or FALSE. Never both.
	\vskip6pt
	\includegraphics[width=1.0\textwidth,trim={0 0.5cm 0 0},clip]{true_false}
	\column{.35\textwidth}
\begin{knitrout}\small
\definecolor{shadecolor}{rgb}{0.969, 0.969, 0.969}\color{fgcolor}\begin{kframe}
\begin{alltt}
\hlstd{> }\hlnum{2} \hlopt{>} \hlnum{1}
\end{alltt}
\begin{verbatim}
[1] TRUE
\end{verbatim}
\begin{alltt}
\hlstd{> }\hlnum{6} \hlopt{<} \hlnum{2}
\end{alltt}
\begin{verbatim}
[1] FALSE
\end{verbatim}
\begin{alltt}
\hlstd{> }\hlnum{1} \hlopt{+} \hlnum{1} \hlopt{==} \hlnum{2}
\end{alltt}
\begin{verbatim}
[1] TRUE
\end{verbatim}
\begin{alltt}
\hlstd{> }\hlnum{6} \hlopt{-} \hlnum{5} \hlopt{==} \hlnum{3}
\end{alltt}
\begin{verbatim}
[1] FALSE
\end{verbatim}
\begin{alltt}
\hlstd{> }\hlnum{10} \hlopt{/} \hlnum{5} \hlopt{==} \hlnum{2}
\end{alltt}
\begin{verbatim}
[1] TRUE
\end{verbatim}
\end{kframe}
\end{knitrout}
	\column{.3\textwidth}
\begin{knitrout}\small
\definecolor{shadecolor}{rgb}{0.969, 0.969, 0.969}\color{fgcolor}\begin{kframe}
\begin{alltt}
\hlstd{> }\hlnum{TRUE} \hlopt{==} \hlnum{FALSE}
\end{alltt}
\begin{verbatim}
[1] FALSE
\end{verbatim}
\begin{alltt}
\hlstd{> }\hlcom{# Not equal}
\hlstd{> }\hlnum{TRUE} \hlopt{!=} \hlnum{FALSE}
\end{alltt}
\begin{verbatim}
[1] TRUE
\end{verbatim}
\begin{alltt}
\hlstd{> }\hlcom{# How about characters?}
\hlstd{> }\hlstr{"Hello"} \hlopt{==} \hlstr{"Hello"}
\end{alltt}
\begin{verbatim}
[1] TRUE
\end{verbatim}
\begin{alltt}
\hlstd{> }\hlstr{"Hello"} \hlopt{==} \hlstr{"hello"}
\end{alltt}
\begin{verbatim}
[1] FALSE
\end{verbatim}
\end{kframe}
\end{knitrout}
    \end{columns}
\end{frame}

\begin{frame}[fragile]
    \frametitle{logical}
    \fontsize{14pt}{5}\selectfont
    \vskip16pt
    You can also use the number form of logicals:\\
    \vskip16pt
    \tt{TRUE} is 1. \tt{FALSE} is 0
    \vskip16pt

\begin{knitrout}\small
\definecolor{shadecolor}{rgb}{0.969, 0.969, 0.969}\color{fgcolor}\begin{kframe}
\begin{alltt}
\hlstd{> }\hlstd{x} \hlkwb{<-} \hlnum{TRUE}
\hlstd{> }\hlnum{3} \hlopt{*} \hlstd{x}
\end{alltt}
\begin{verbatim}
[1] 3
\end{verbatim}
\begin{alltt}
\hlstd{> }\hlstd{y} \hlkwb{<-} \hlnum{FALSE}
\hlstd{> }\hlnum{123} \hlopt{*} \hlstd{y}
\end{alltt}
\begin{verbatim}
[1] 0
\end{verbatim}
\begin{alltt}
\hlstd{> }\hlstd{x} \hlopt{+} \hlstd{y}
\end{alltt}
\begin{verbatim}
[1] 1
\end{verbatim}
\begin{alltt}
\hlstd{> }\hlstd{(}\hlnum{3} \hlopt{*} \hlstd{x)} \hlopt{+} \hlstd{(}\hlnum{123} \hlopt{*} \hlstd{y)}
\end{alltt}
\begin{verbatim}
[1] 3
\end{verbatim}
\end{kframe}
\end{knitrout}

\end{frame}

\begin{frame}[fragile]
    \frametitle{logical}
    \begin{columns}[t]
	\column{.4\textwidth}
	\tt{TRUE} and \tt{FALSE} are protected in R i.e. you can't modify them.
	They can be a bit of a pain to type. You can be naughty and use
	shortcut T and F. \alert{Don't} because...

	\vskip6pt
	\includegraphics[width=1.0\textwidth,trim={0 1.5cm 0 0},clip]{true_false2}
	\column{.2\textwidth}
\begin{knitrout}\small
\definecolor{shadecolor}{rgb}{0.969, 0.969, 0.969}\color{fgcolor}\begin{kframe}
\begin{alltt}
\hlstd{> }\hlstd{T}
\end{alltt}
\begin{verbatim}
[1] TRUE
\end{verbatim}
\begin{alltt}
\hlstd{> }\hlstd{F}
\end{alltt}
\begin{verbatim}
[1] FALSE
\end{verbatim}
\begin{alltt}
\hlstd{> }\hlstd{x} \hlkwb{<-} \hlstd{T}
\hlstd{> }\hlstd{x}
\end{alltt}
\begin{verbatim}
[1] TRUE
\end{verbatim}
\begin{alltt}
\hlstd{> }\hlstd{T} \hlkwb{<-} \hlnum{FALSE}
\hlstd{> }\hlstd{T}
\end{alltt}
\begin{verbatim}
[1] FALSE
\end{verbatim}
\begin{alltt}
\hlstd{> }\hlstd{x} \hlkwb{<-} \hlstd{T}
\hlstd{> }\hlstd{x}
\end{alltt}
\begin{verbatim}
[1] FALSE
\end{verbatim}
\end{kframe}
\end{knitrout}


	\column{.4\textwidth}
\begin{knitrout}\small
\definecolor{shadecolor}{rgb}{0.969, 0.969, 0.969}\color{fgcolor}\begin{kframe}
\begin{alltt}
\hlstd{> }\hlcom{# valid}
\hlstd{> }\hlstd{T} \hlkwb{<-} \hlstr{"Thank You"}
\hlstd{> }\hlstd{T}
\end{alltt}
\begin{verbatim}
[1] "Thank You"
\end{verbatim}
\begin{alltt}
\hlstd{> }\hlstd{x} \hlkwb{<-} \hlstd{T}
\hlstd{> }\hlstd{x}
\end{alltt}
\begin{verbatim}
[1] "Thank You"
\end{verbatim}
\begin{alltt}
\hlstd{> }\hlcom{# invalid}
\hlstd{> }\hlnum{TRUE} \hlkwb{<-} \hlstr{"You're Welcome"}
\end{alltt}


{\ttfamily\noindent\bfseries\color{errorcolor}{Error in TRUE <- "{}You're Welcome"{}: invalid (do\_set) left-hand side to assignment}}\begin{alltt}
\hlstd{> }\hlnum{TRUE}
\end{alltt}
\begin{verbatim}
[1] TRUE
\end{verbatim}
\end{kframe}
\end{knitrout}
    \end{columns}
\end{frame}

\begin{frame}[fragile]
    \frametitle{factor}
When you have only a few different "categories" of data values
\vskip6pt
e.g. gender, country, age group, religion etc.
\vskip6pt
\begin{knitrout}\small
\definecolor{shadecolor}{rgb}{0.969, 0.969, 0.969}\color{fgcolor}\begin{kframe}
\begin{alltt}
\hlstd{> }\hlstd{blood} \hlkwb{<-} \hlkwd{c}\hlstd{(}\hlstr{"B"}\hlstd{,} \hlstr{"AB"}\hlstd{,} \hlstr{"O"}\hlstd{,} \hlstr{"A"}\hlstd{,} \hlstr{"O"}\hlstd{,} \hlstr{"O"}\hlstd{,} \hlstr{"A"}\hlstd{,} \hlstr{"B"}\hlstd{)}
\hlstd{> }\hlstd{blood}
\end{alltt}
\begin{verbatim}
[1] "B"  "AB" "O"  "A"  "O"  "O"  "A"  "B" 
\end{verbatim}
\begin{alltt}
\hlstd{> }\hlstd{blood_factor} \hlkwb{<-} \hlkwd{factor}\hlstd{(blood)}
\hlstd{> }\hlstd{blood_factor}
\end{alltt}
\begin{verbatim}
[1] B  AB O  A  O  O  A  B 
Levels: A AB B O
\end{verbatim}
\begin{alltt}
\hlstd{> }\hlkwd{str}\hlstd{(blood_factor)}
\end{alltt}
\begin{verbatim}
 Factor w/ 4 levels "A","AB","B","O": 3 2 4 1 4 4 1 3
\end{verbatim}
\end{kframe}
\end{knitrout}
\end{frame}

\begin{frame}[fragile]
    \frametitle{function \tt{is}...}
    Helpful to check if something \alert{\textit{is something}}:\\
\begin{knitrout}\small
\definecolor{shadecolor}{rgb}{0.969, 0.969, 0.969}\color{fgcolor}\begin{kframe}
\begin{alltt}
\hlstd{> }\hlkwd{is.numeric}\hlstd{(}\hlnum{13}\hlstd{)}
\end{alltt}
\begin{verbatim}
[1] TRUE
\end{verbatim}
\begin{alltt}
\hlstd{> }\hlkwd{is.numeric}\hlstd{(}\hlopt{-}\hlnum{1}\hlstd{)}
\end{alltt}
\begin{verbatim}
[1] TRUE
\end{verbatim}
\begin{alltt}
\hlstd{> }\hlkwd{is.numeric}\hlstd{(}\hlnum{0.0004}\hlstd{)}
\end{alltt}
\begin{verbatim}
[1] TRUE
\end{verbatim}
\begin{alltt}
\hlstd{> }\hlkwd{is.numeric}\hlstd{(}\hlstr{"This is stupid"}\hlstd{)}
\end{alltt}
\begin{verbatim}
[1] FALSE
\end{verbatim}
\begin{alltt}
\hlstd{> }\hlkwd{is.numeric}\hlstd{(}\hlnum{100}\hlstd{,}\hlnum{000}\hlstd{)} \hlcom{# can't use thousand separator!}
\end{alltt}


{\ttfamily\noindent\bfseries\color{errorcolor}{Error in is.numeric(100, 0): 2 arguments passed to 'is.numeric' which requires 1}}\begin{alltt}
\hlstd{> }\hlkwd{is.numeric}\hlstd{(}\hlnum{100000}\hlstd{)}
\end{alltt}
\begin{verbatim}
[1] TRUE
\end{verbatim}
\end{kframe}
\end{knitrout}
\end{frame}

\begin{frame}[fragile]
    \frametitle{function \tt{is}...}
    Helpful to check if something \alert{\textit{is something}}:\\
\begin{knitrout}\small
\definecolor{shadecolor}{rgb}{0.969, 0.969, 0.969}\color{fgcolor}\begin{kframe}
\begin{alltt}
\hlstd{> }\hlstd{et} \hlkwb{<-} \hlstr{"E.T. Go Home"}
\hlstd{> }\hlstd{darth_vader} \hlkwb{<-} \hlstr{"Luke I'm your father"}
\hlstd{> }\hlstd{num_but_char} \hlkwb{<-} \hlstr{"13"}
\hlstd{> }\hlkwd{is.character}\hlstd{(et)}
\end{alltt}
\begin{verbatim}
[1] TRUE
\end{verbatim}
\begin{alltt}
\hlstd{> }\hlkwd{is.character}\hlstd{(darth_vader)}
\end{alltt}
\begin{verbatim}
[1] TRUE
\end{verbatim}
\begin{alltt}
\hlstd{> }\hlkwd{is.character}\hlstd{(num_but_char)}
\end{alltt}
\begin{verbatim}
[1] TRUE
\end{verbatim}
\begin{alltt}
\hlstd{> }\hlkwd{is.numeric}\hlstd{(et)}
\end{alltt}
\begin{verbatim}
[1] FALSE
\end{verbatim}
\begin{alltt}
\hlstd{> }\hlkwd{is.numeric}\hlstd{(num_but_char)}
\end{alltt}
\begin{verbatim}
[1] FALSE
\end{verbatim}
\end{kframe}
\end{knitrout}
\end{frame}

\begin{frame}[fragile]
    \frametitle{function \tt{as}...}
    Helpful to convert (coerce) something \alert{\textit{as something}}:\\
    \begin{columns}[t]
	\column{0.5\textwidth}
\begin{knitrout}\small
\definecolor{shadecolor}{rgb}{0.969, 0.969, 0.969}\color{fgcolor}\begin{kframe}
\begin{alltt}
\hlstd{> }\hlkwd{as.numeric}\hlstd{(}\hlnum{TRUE}\hlstd{)}
\end{alltt}
\begin{verbatim}
[1] 1
\end{verbatim}
\begin{alltt}
\hlstd{> }\hlkwd{as.numeric}\hlstd{(}\hlnum{FALSE}\hlstd{)}
\end{alltt}
\begin{verbatim}
[1] 0
\end{verbatim}
\begin{alltt}
\hlstd{> }\hlkwd{as.character}\hlstd{(}\hlnum{4}\hlstd{)}
\end{alltt}
\begin{verbatim}
[1] "4"
\end{verbatim}
\begin{alltt}
\hlstd{> }\hlkwd{as.numeric}\hlstd{(}\hlstr{"4.5"}\hlstd{)}
\end{alltt}
\begin{verbatim}
[1] 4.5
\end{verbatim}
\begin{alltt}
\hlstd{> }\hlcom{# Can't coerce characters to}
\hlstd{> }\hlcom{# numeric, unless they're quoted}
\hlstd{> }\hlcom{# numbers.}
\hlstd{> }\hlkwd{as.numeric}\hlstd{(}\hlstr{"Hello"}\hlstd{)}
\end{alltt}


{\ttfamily\noindent\color{warningcolor}{Warning: NAs introduced by coercion}}\begin{verbatim}
[1] NA
\end{verbatim}
\end{kframe}
\end{knitrout}
	\column{0.5\textwidth}
\begin{knitrout}\small
\definecolor{shadecolor}{rgb}{0.969, 0.969, 0.969}\color{fgcolor}\begin{kframe}
\begin{alltt}
\hlstd{> }\hlcom{# All numbers are TRUE, except 0}
\hlstd{> }\hlkwd{as.logical}\hlstd{(}\hlnum{0}\hlstd{)}
\end{alltt}
\begin{verbatim}
[1] FALSE
\end{verbatim}
\begin{alltt}
\hlstd{> }\hlkwd{as.logical}\hlstd{(}\hlopt{-}\hlnum{5}\hlstd{)}
\end{alltt}
\begin{verbatim}
[1] TRUE
\end{verbatim}
\begin{alltt}
\hlstd{> }\hlkwd{as.logical}\hlstd{(}\hlnum{1}\hlstd{)}
\end{alltt}
\begin{verbatim}
[1] TRUE
\end{verbatim}
\begin{alltt}
\hlstd{> }\hlkwd{as.logical}\hlstd{(}\hlnum{5.9}\hlstd{)}
\end{alltt}
\begin{verbatim}
[1] TRUE
\end{verbatim}
\end{kframe}
\end{knitrout}
    \end{columns}
\end{frame}

\begin{frame}
    \begin{center}
	\textbf{\Huge{\tc{Blue}{Vectors and Subsetting}}}
    \end{center}
\end{frame}

\begin{frame}
    \frametitle{Vector}
    \begin{itemize}
	\item Sequence of data elements
	\item Same basic type
	\item \tt{numeric}, \tt{logical}, \tt{character}
    \end{itemize}
\end{frame}

\begin{frame}[fragile]
    \frametitle{Create a vector with \tt{c()}}
\begin{knitrout}\small
\definecolor{shadecolor}{rgb}{0.969, 0.969, 0.969}\color{fgcolor}\begin{kframe}
\begin{alltt}
\hlstd{> }\hlstd{fruit} \hlkwb{<-} \hlkwd{c}\hlstd{(}\hlstr{"banana"}\hlstd{,} \hlstr{"apple"}\hlstd{,} \hlstr{"apple"}\hlstd{,} \hlstr{"kiwi"}\hlstd{,} \hlstr{"pineapple"}\hlstd{,} \hlstr{"strawberry"}\hlstd{)}
\hlstd{> }\hlstd{fruit}
\end{alltt}
\begin{verbatim}
[1] "banana"     "apple"      "apple"      "kiwi"       "pineapple" 
[6] "strawberry"
\end{verbatim}
\begin{alltt}
\hlstd{> }\hlstd{drawn_suits} \hlkwb{<-} \hlkwd{c}\hlstd{(}\hlstr{"hearts"}\hlstd{,} \hlstr{"spades"}\hlstd{,} \hlstr{"diamonds"}\hlstd{,} \hlstr{"diamonds"}\hlstd{,} \hlstr{"spades"}\hlstd{)}
\hlstd{> }\hlstd{drawn_suits}
\end{alltt}
\begin{verbatim}
[1] "hearts"   "spades"   "diamonds" "diamonds" "spades"  
\end{verbatim}
\begin{alltt}
\hlstd{> }\hlstd{lucky_numbers} \hlkwb{<-} \hlkwd{c}\hlstd{(}\hlnum{10}\hlstd{,} \hlnum{7}\hlstd{,} \hlnum{13}\hlstd{,} \hlnum{42}\hlstd{,} \hlnum{99}\hlstd{)}
\hlstd{> }\hlstd{lucky_numbers}
\end{alltt}
\begin{verbatim}
[1] 10  7 13 42 99
\end{verbatim}
\begin{alltt}
\hlstd{> }\hlkwd{is.vector}\hlstd{(drawn_suits)}
\end{alltt}
\begin{verbatim}
[1] TRUE
\end{verbatim}
\begin{alltt}
\hlstd{> }\hlkwd{is.vector}\hlstd{(fruit)}
\end{alltt}
\begin{verbatim}
[1] TRUE
\end{verbatim}
\begin{alltt}
\hlstd{> }\hlkwd{is.vector}\hlstd{(lucky_numbers)}
\end{alltt}
\begin{verbatim}
[1] TRUE
\end{verbatim}
\end{kframe}
\end{knitrout}
\end{frame}

\begin{frame}[fragile]
    \frametitle{Create a vector with \tt{:}}
\begin{knitrout}\small
\definecolor{shadecolor}{rgb}{0.969, 0.969, 0.969}\color{fgcolor}\begin{kframe}
\begin{alltt}
\hlstd{> }\hlstd{quick_vec} \hlkwb{<-} \hlnum{1}\hlopt{:}\hlnum{10}
\hlstd{> }\hlstd{quick_vec}
\end{alltt}
\begin{verbatim}
 [1]  1  2  3  4  5  6  7  8  9 10
\end{verbatim}
\begin{alltt}
\hlstd{> }\hlstd{x} \hlkwb{<-} \hlnum{5}\hlopt{:}\hlnum{12}
\hlstd{> }\hlstd{x}
\end{alltt}
\begin{verbatim}
[1]  5  6  7  8  9 10 11 12
\end{verbatim}
\begin{alltt}
\hlstd{> }\hlstd{y} \hlkwb{<-} \hlopt{-}\hlnum{2}\hlopt{:}\hlnum{3}
\hlstd{> }\hlstd{y}
\end{alltt}
\begin{verbatim}
[1] -2 -1  0  1  2  3
\end{verbatim}
\begin{alltt}
\hlstd{> }\hlstd{z} \hlkwb{<-} \hlnum{3.2}\hlopt{:}\hlnum{10}
\hlstd{> }\hlstd{z}
\end{alltt}
\begin{verbatim}
[1] 3.2 4.2 5.2 6.2 7.2 8.2 9.2
\end{verbatim}
\end{kframe}
\end{knitrout}
\end{frame}

\begin{frame}[fragile]
    \frametitle{Get a vector's length with... \tt{length()}}
\begin{knitrout}\small
\definecolor{shadecolor}{rgb}{0.969, 0.969, 0.969}\color{fgcolor}\begin{kframe}
\begin{alltt}
\hlstd{> }\hlstd{fruit}
\end{alltt}
\begin{verbatim}
[1] "banana"     "apple"      "apple"      "kiwi"       "pineapple" 
[6] "strawberry"
\end{verbatim}
\begin{alltt}
\hlstd{> }\hlkwd{length}\hlstd{(fruit)}
\end{alltt}
\begin{verbatim}
[1] 6
\end{verbatim}
\begin{alltt}
\hlstd{> }\hlstd{drawn_suits}
\end{alltt}
\begin{verbatim}
[1] "hearts"   "spades"   "diamonds" "diamonds" "spades"  
\end{verbatim}
\begin{alltt}
\hlstd{> }\hlkwd{length}\hlstd{(lucky_numbers)}
\end{alltt}
\begin{verbatim}
[1] 5
\end{verbatim}
\begin{alltt}
\hlstd{> }\hlkwd{length}\hlstd{(}\hlnum{10}\hlstd{)}
\end{alltt}
\begin{verbatim}
[1] 1
\end{verbatim}
\begin{alltt}
\hlstd{> }\hlkwd{length}\hlstd{(}\hlkwd{c}\hlstd{(}\hlnum{3}\hlstd{))}
\end{alltt}
\begin{verbatim}
[1] 1
\end{verbatim}
\begin{alltt}
\hlstd{> }\hlkwd{length}\hlstd{(}\hlkwd{c}\hlstd{())} \hlcom{# empty vector}
\end{alltt}
\begin{verbatim}
[1] 0
\end{verbatim}
\end{kframe}
\end{knitrout}

\end{frame}

\begin{frame}[fragile]
    \frametitle{Coercion for vectors}
\begin{knitrout}\small
\definecolor{shadecolor}{rgb}{0.969, 0.969, 0.969}\color{fgcolor}\begin{kframe}
\begin{alltt}
\hlstd{> }\hlstd{drawn_ranks} \hlkwb{<-} \hlkwd{c}\hlstd{(}\hlnum{7}\hlstd{,} \hlnum{4}\hlstd{,} \hlstr{"A"}\hlstd{,} \hlnum{10}\hlstd{,} \hlstr{"K"}\hlstd{,} \hlnum{3}\hlstd{,} \hlnum{2}\hlstd{,} \hlstr{"Q"}\hlstd{)}
\hlstd{> }\hlstd{drawn_ranks}
\end{alltt}
\begin{verbatim}
[1] "7"  "4"  "A"  "10" "K"  "3"  "2"  "Q" 
\end{verbatim}
\begin{alltt}
\hlstd{> }\hlkwd{str}\hlstd{(drawn_ranks)}
\end{alltt}
\begin{verbatim}
 chr [1:8] "7" "4" "A" "10" "K" "3" "2" "Q"
\end{verbatim}
\end{kframe}
\end{knitrout}

\end{frame}

\begin{frame}[fragile]
    \frametitle{Vector Arithmetic}
\vskip10pt
Single element arithmetic
\vskip10pt
\begin{knitrout}\small
\definecolor{shadecolor}{rgb}{0.969, 0.969, 0.969}\color{fgcolor}\begin{kframe}
\begin{alltt}
\hlstd{> }\hlstd{my_apples} \hlkwb{<-} \hlnum{5}
\hlstd{> }\hlstd{my_oranges} \hlkwb{<-} \hlnum{6}
\hlstd{> }\hlstd{my_apples} \hlopt{+} \hlstd{my_oranges}
\end{alltt}
\begin{verbatim}
[1] 11
\end{verbatim}
\begin{alltt}
\hlstd{> }\hlstd{my_apples} \hlopt{-} \hlstd{my_oranges}
\end{alltt}
\begin{verbatim}
[1] -1
\end{verbatim}
\begin{alltt}
\hlstd{> }\hlstd{my_apples} \hlopt{*} \hlstd{my_oranges}
\end{alltt}
\begin{verbatim}
[1] 30
\end{verbatim}
\begin{alltt}
\hlstd{> }\hlstd{my_apples} \hlopt{/} \hlstd{my_oranges}
\end{alltt}
\begin{verbatim}
[1] 0.8333333
\end{verbatim}
\end{kframe}
\end{knitrout}
\end{frame}

\begin{frame}[fragile]
    \frametitle{Vector Arithmetic}
\vskip10pt
Computations performed element-wise
\vskip10pt
\begin{knitrout}\small
\definecolor{shadecolor}{rgb}{0.969, 0.969, 0.969}\color{fgcolor}\begin{kframe}
\begin{alltt}
\hlstd{> }\hlstd{money} \hlkwb{<-} \hlkwd{c}\hlstd{(}\hlnum{50}\hlstd{,} \hlnum{100}\hlstd{,} \hlnum{40}\hlstd{)}
\hlstd{> }\hlstd{money} \hlopt{*} \hlnum{2}
\end{alltt}
\begin{verbatim}
[1] 100 200  80
\end{verbatim}
\begin{alltt}
\hlstd{> }\hlstd{money} \hlopt{/} \hlnum{10}
\end{alltt}
\begin{verbatim}
[1]  5 10  4
\end{verbatim}
\begin{alltt}
\hlstd{> }\hlstd{money} \hlopt{+} \hlnum{10}
\end{alltt}
\begin{verbatim}
[1]  60 110  50
\end{verbatim}
\begin{alltt}
\hlstd{> }\hlstd{money} \hlopt{^} \hlnum{2}
\end{alltt}
\begin{verbatim}
[1]  2500 10000  1600
\end{verbatim}
\end{kframe}
\end{knitrout}
\end{frame}

\begin{frame}[fragile]
    \frametitle{Vector Arithmetic}
\vskip10pt
\tt{sum()} of all elements
\vskip10pt
\begin{knitrout}\small
\definecolor{shadecolor}{rgb}{0.969, 0.969, 0.969}\color{fgcolor}\begin{kframe}
\begin{alltt}
\hlstd{> }\hlstd{money}
\end{alltt}
\begin{verbatim}
[1]  50 100  40
\end{verbatim}
\begin{alltt}
\hlstd{> }\hlkwd{sum}\hlstd{(money)}
\end{alltt}
\begin{verbatim}
[1] 190
\end{verbatim}
\begin{alltt}
\hlstd{> }\hlkwd{sum}\hlstd{(}\hlkwd{c}\hlstd{(}\hlnum{0}\hlstd{,} \hlnum{0}\hlstd{,} \hlopt{-}\hlnum{5}\hlstd{,} \hlnum{4}\hlstd{))}
\end{alltt}
\begin{verbatim}
[1] -1
\end{verbatim}
\begin{alltt}
\hlstd{> }\hlkwd{sum}\hlstd{(}\hlkwd{c}\hlstd{(}\hlnum{FALSE}\hlstd{,} \hlnum{TRUE}\hlstd{,} \hlnum{FALSE}\hlstd{,} \hlnum{TRUE}\hlstd{,} \hlnum{TRUE}\hlstd{))} \hlcom{# very handy}
\end{alltt}
\begin{verbatim}
[1] 3
\end{verbatim}
\begin{alltt}
\hlstd{> }\hlkwd{sum}\hlstd{(}\hlkwd{c}\hlstd{(}\hlnum{FALSE}\hlstd{,} \hlnum{TRUE}\hlstd{,} \hlnum{NA}\hlstd{,} \hlnum{FALSE}\hlstd{))} \hlcom{# NA: missing data}
\end{alltt}
\begin{verbatim}
[1] NA
\end{verbatim}
\begin{alltt}
\hlstd{> }\hlkwd{sum}\hlstd{(}\hlkwd{c}\hlstd{(}\hlnum{FALSE}\hlstd{,} \hlnum{TRUE}\hlstd{,} \hlnum{NA}\hlstd{,} \hlnum{FALSE}\hlstd{),} \hlkwc{na.rm} \hlstd{=} \hlnum{TRUE}\hlstd{)} \hlcom{# ignore NAs}
\end{alltt}
\begin{verbatim}
[1] 1
\end{verbatim}
\end{kframe}
\end{knitrout}
\end{frame}

\begin{frame}[fragile]
    \frametitle{Vector Arithmetic}
\vskip10pt
\tt{sum()} of all elements
\vskip10pt
\begin{knitrout}\small
\definecolor{shadecolor}{rgb}{0.969, 0.969, 0.969}\color{fgcolor}\begin{kframe}
\begin{alltt}
\hlstd{> }\hlstd{a} \hlkwb{<-} \hlnum{1}\hlopt{:}\hlnum{10}\hlstd{; a}
\end{alltt}
\begin{verbatim}
 [1]  1  2  3  4  5  6  7  8  9 10
\end{verbatim}
\begin{alltt}
\hlstd{> }\hlstd{b} \hlkwb{<-} \hlkwd{c}\hlstd{(}\hlnum{10}\hlstd{,} \hlnum{20}\hlstd{,} \hlnum{30}\hlstd{,} \hlnum{40}\hlstd{,} \hlnum{50}\hlstd{); b}
\end{alltt}
\begin{verbatim}
[1] 10 20 30 40 50
\end{verbatim}
\begin{alltt}
\hlstd{> }\hlkwd{sum}\hlstd{(a)}
\end{alltt}
\begin{verbatim}
[1] 55
\end{verbatim}
\begin{alltt}
\hlstd{> }\hlkwd{sum}\hlstd{(b)}
\end{alltt}
\begin{verbatim}
[1] 150
\end{verbatim}
\begin{alltt}
\hlstd{> }\hlkwd{mean}\hlstd{(a)}
\end{alltt}
\begin{verbatim}
[1] 5.5
\end{verbatim}
\begin{alltt}
\hlstd{> }\hlkwd{sqrt}\hlstd{(b)}
\end{alltt}
\begin{verbatim}
[1] 3.162278 4.472136 5.477226 6.324555 7.071068
\end{verbatim}
\end{kframe}
\end{knitrout}
\end{frame}

\begin{frame}[fragile]
    \frametitle{Comparing vectors}
\begin{knitrout}\small
\definecolor{shadecolor}{rgb}{0.969, 0.969, 0.969}\color{fgcolor}\begin{kframe}
\begin{alltt}
\hlstd{> }\hlstd{earnings} \hlkwb{<-} \hlkwd{c}\hlstd{(}\hlnum{50}\hlstd{,} \hlnum{100}\hlstd{,} \hlnum{30}\hlstd{)}
\hlstd{> }\hlstd{expenses} \hlkwb{<-} \hlkwd{c}\hlstd{(}\hlnum{10}\hlstd{,} \hlnum{20}\hlstd{,} \hlnum{30}\hlstd{)}
\hlstd{> }\hlstd{bank} \hlkwb{<-} \hlstd{earnings} \hlopt{-} \hlstd{expenses}
\hlstd{> }\hlstd{bank}
\end{alltt}
\begin{verbatim}
[1] 40 80  0
\end{verbatim}
\begin{alltt}
\hlstd{> }\hlkwd{sum}\hlstd{(bank)}
\end{alltt}
\begin{verbatim}
[1] 120
\end{verbatim}
\begin{alltt}
\hlstd{> }\hlstd{earnings} \hlopt{>} \hlstd{expenses}
\end{alltt}
\begin{verbatim}
[1]  TRUE  TRUE FALSE
\end{verbatim}
\begin{alltt}
\hlstd{> }\hlstd{earnings} \hlopt{==} \hlstd{expenses}
\end{alltt}
\begin{verbatim}
[1] FALSE FALSE  TRUE
\end{verbatim}
\end{kframe}
\end{knitrout}
\end{frame}

\begin{frame}[fragile]
    \frametitle{Subset vectors - the mighty \tt{[}}
\vskip10pt
By index
\vskip10pt
\begin{knitrout}\small
\definecolor{shadecolor}{rgb}{0.969, 0.969, 0.969}\color{fgcolor}\begin{kframe}
\begin{alltt}
\hlstd{> }\hlstd{x1} \hlkwb{<-} \hlkwd{c}\hlstd{(}\hlnum{10}\hlstd{,} \hlnum{5}\hlstd{,} \hlnum{3}\hlstd{,} \hlopt{-}\hlnum{12}\hlstd{,} \hlnum{19}\hlstd{,} \hlnum{145}\hlstd{)}
\hlstd{> }\hlstd{x1[}\hlnum{1}\hlstd{]} \hlcom{# first element has index 1}
\end{alltt}
\begin{verbatim}
[1] 10
\end{verbatim}
\begin{alltt}
\hlstd{> }\hlstd{n} \hlkwb{<-} \hlkwd{length}\hlstd{(x1)} \hlcom{# last element has index the length of the vec}
\hlstd{> }\hlstd{n}
\end{alltt}
\begin{verbatim}
[1] 6
\end{verbatim}
\begin{alltt}
\hlstd{> }\hlstd{x1[n]}
\end{alltt}
\begin{verbatim}
[1] 145
\end{verbatim}
\begin{alltt}
\hlstd{> }\hlstd{x1[}\hlnum{10}\hlstd{]} \hlcom{# only n elements!}
\end{alltt}
\begin{verbatim}
[1] NA
\end{verbatim}
\end{kframe}
\end{knitrout}
\end{frame}

\begin{frame}[fragile]
    \frametitle{Subset vectors}
\begin{knitrout}\small
\definecolor{shadecolor}{rgb}{0.969, 0.969, 0.969}\color{fgcolor}\begin{kframe}
\begin{alltt}
\hlstd{> }\hlstd{x1}
\end{alltt}
\begin{verbatim}
[1]  10   5   3 -12  19 145
\end{verbatim}
\begin{alltt}
\hlstd{> }\hlstd{x1[}\hlopt{-}\hlnum{1}\hlstd{]} \hlcom{# all but the first element}
\end{alltt}
\begin{verbatim}
[1]   5   3 -12  19 145
\end{verbatim}
\begin{alltt}
\hlstd{> }\hlstd{x1[}\hlkwd{c}\hlstd{(}\hlnum{3}\hlstd{,} \hlnum{5}\hlstd{)]}
\end{alltt}
\begin{verbatim}
[1]  3 19
\end{verbatim}
\begin{alltt}
\hlstd{> }\hlstd{x1[}\hlopt{-}\hlkwd{c}\hlstd{(}\hlnum{1}\hlstd{,} \hlnum{6}\hlstd{)]}
\end{alltt}
\begin{verbatim}
[1]   5   3 -12  19
\end{verbatim}
\begin{alltt}
\hlstd{> }\hlstd{x1[}\hlkwd{c}\hlstd{(}\hlnum{TRUE}\hlstd{,} \hlnum{TRUE}\hlstd{,} \hlnum{FALSE}\hlstd{,} \hlnum{FALSE}\hlstd{,} \hlnum{TRUE}\hlstd{,} \hlnum{FALSE}\hlstd{)]}
\end{alltt}
\begin{verbatim}
[1] 10  5 19
\end{verbatim}
\begin{alltt}
\hlstd{> }\hlstd{x1[}\hlkwd{c}\hlstd{(}   \hlnum{1}\hlstd{,}    \hlnum{2}\hlstd{,}                  \hlnum{5}\hlstd{)]} \hlcom{# spaces are ignored}
\end{alltt}
\begin{verbatim}
[1] 10  5 19
\end{verbatim}
\begin{alltt}
\hlstd{> }\hlstd{x1[}\hlkwd{c}\hlstd{(}\hlnum{TRUE}\hlstd{,} \hlnum{TRUE}\hlstd{,} \hlnum{FALSE}\hlstd{)]} \hlcom{# recycle when less than n!}
\end{alltt}
\begin{verbatim}
[1]  10   5 -12  19
\end{verbatim}
\end{kframe}
\end{knitrout}
\end{frame}

\begin{frame}[fragile]
    \frametitle{Subset vectors}
\begin{knitrout}\small
\definecolor{shadecolor}{rgb}{0.969, 0.969, 0.969}\color{fgcolor}\begin{kframe}
\begin{alltt}
\hlstd{> }\hlcom{#?rnorm}
\hlstd{> }\hlstd{x} \hlkwb{<-} \hlkwd{rnorm}\hlstd{(}\hlnum{10}\hlstd{)}
\hlstd{> }\hlstd{x}
\end{alltt}
\begin{verbatim}
 [1]  0.96431275 -0.20086165  1.51517615  0.23079883 -0.01346703
 [6] -1.19497238  0.55480499  0.92617387 -1.07999909  0.52331375
\end{verbatim}
\begin{alltt}
\hlstd{> }\hlstd{x} \hlopt{>} \hlnum{0}
\end{alltt}
\begin{verbatim}
 [1]  TRUE FALSE  TRUE  TRUE FALSE FALSE  TRUE  TRUE FALSE  TRUE
\end{verbatim}
\begin{alltt}
\hlstd{> }\hlkwd{sum}\hlstd{(x} \hlopt{>} \hlnum{0}\hlstd{)}
\end{alltt}
\begin{verbatim}
[1] 6
\end{verbatim}
\begin{alltt}
\hlstd{> }\hlstd{x[x} \hlopt{>} \hlnum{0}\hlstd{]}
\end{alltt}
\begin{verbatim}
[1] 0.9643127 1.5151762 0.2307988 0.5548050 0.9261739 0.5233137
\end{verbatim}
\end{kframe}
\end{knitrout}
\end{frame}

\end{document}
